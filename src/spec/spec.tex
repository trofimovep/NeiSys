\documentclass{article}
\usepackage[12pt]{extsizes}
\usepackage{cmap}
\usepackage[utf8]{inputenc}
%\usepackage[cp1251]{inputenc}
\usepackage[T2A]{fontenc}
\usepackage[russian]{babel}
\usepackage{color}
\usepackage{hyperref}
\usepackage{enumerate}
\usepackage{amsmath,amssymb, wasysym}
\usepackage{enumitem}
\usepackage{epstopdf}
\usepackage{graphicx}
\usepackage[warn]{mathtext} 

%\graphicspath{{noiseimages/}}
\hypersetup{unicode = true, colorlinks, citecolor = red, filecolor = red, linkcolor = blue, urlcolor = blue }

\usepackage[left=2cm,right=2cm,
    top=2cm,bottom=2cm,bindingoffset=0cm]{geometry}


\begin{document}
\pagestyle{empty}

It is needed to do:
\begin{enumerate}
\item{.} Menu:
\subitem{..} Save model
\subitem{..} Load model
\subitem{..} Delete model

\item{.}  Scroll Panel

Параметр: $OPTION_SIZE$ -- задает размерность для параметров узлов(например, $OPTION_SIZE = 2$-- параметры для узлов - матрицы, для $OPTION_SIZE = 3$ какие-то трехмерные матрицы (но не тензоры!)).

\paragraph{trofimovep.Knot}
Всего два типа узлов: Состояние (trofimovep.trofimovep.State) и Управление (trofimovep.trofimovep.Control).


\paragraph{Relations} 
Для Отношений введены запреты на 
\begin{enumerate}
\item На связь Управление-Управление (функция isCorrect)
\item Связь Управление-Состояние (нельзя состоянием повлиять на управление) (функция isCorrect)
\item Самосвязь (узел не может воздействовать сам на себя) за это отвечает цикл класса trofimovep.StartWindow
в MouseReleased (при {\it{if (id1 == id2)}} ничего не происходит):
\begin{verbatim}
 if (id1 == id2) {
   }
    else if (isCorrect(id1, id2, knots) == true)
         relations.add(new trofimovep.Relation(id1, id2, knots));
    else {
         JOptionPane.showMessageDialog(null, "The operation is forbidden!");
         }
\end{verbatim}

Добавлена \href{http://javadox.com/com.googlecode.efficient-java-matrix-library/ejml/0.23/org/ejml/alg/dense/linsol/svd/SolvePseudoInverseSvd.html}{библиотека для работы с матрицами}, в частности реализовано псевдообращение.

Хорошая идея добавить к параметрам узла какой-либо логический параметр. В качестве разрешающего или запрещающего использование.

\end{enumerate}



\end{enumerate}



\end{document}